\documentclass[12pt]{article}
\usepackage{hyperref}
\usepackage{amsmath}

\newcommand{\vect}[1]{\mathbf{#1}}

% document specific commands
\newcommand{\detgam}{|\gamma|}
\newcommand{\sqdetgam}{\sqrt{|\gamma|}}

\newcommand{\Ahat}{\hat{\vect{A}}}
\newcommand{\Atissue}{\vect{A}_{\text{tissue}}}
\newcommand{\Awater}{\vect{A}_{\text{water}}}
\newcommand{\Abitensor}{\vect{A}_{\text{bi-tensor}}}

\newcommand{\vx}{\vect{x}}
\newcommand{\pp}[1]{\frac{\partial}{\partial #1}}

\title{Single Shell Free Water Elimination Model Implementation
}
\author{
  Sameer D'Costa \\
    sameerdcosta@gmail.com
}
\date{\today}


\begin{document}
\maketitle

\section{Introduction}
This document details out an implementation of a Free Water Elimination model
for Diffusion Tensor MRI. We are mainly following \cite{Pasternak2009} with a
few simplifications suggested in \cite{Pasternak2014}.

\section{Bi-tensor Model}
In this section we summarize the \textbf{Theory} section in
\cite{Pasternak2009}. 

\ 

\noindent
For each voxel we have several readings. $\vect{S}_0$ is the signal acquired
for the zero diffusion weighting from the $\vect{b0}$ image. $\vect{S}_k$ is
the signal from the diffusion weighted image (DWI) when the gradient
orientation $\vect{q}_k$ is applied. We calculate the attenuation
$[\hat{\vect{A}}]_k = \vect{S}_k / \vect{S}_0$. The attenuation values are
between 0 and 1. 

\ 

\noindent
For the \textbf{single compartment model} we assume that the attenuation
$[\Ahat]_k$ all comes from tissue $[\Atissue]_k$. For a Diffusion Tensor
$\vect{D}$ this gives us 
$$[\Atissue(\vect{D})]_k = \exp(-b \vect{q}_k^T\vect{D}\vect{q}_k).$$
The $b$ value in the equation is above is the b-value of our single shell. The
gradient $\vect{q}_k$ is a vector of length 3 and $\vect{D}$ is a 3x3 symmetric
matrix at each voxel. Let's denote by $n$ the number of applied gradients.  The
minimum $n$ required is 6 as $\vect{D}$ has 6 parameters that need to be
estimated at each voxel. Usually we have a large number (somewhere greater than 30 is customary).


\ 

\noindent
If we had water instead of tissue at the voxels then the DTI model (above)
simplifies and we get the same result in all directions. The matrix $\vect{D}$
becomes a scalar $d$ and we get the same value of attenuation at every voxel.
$$[\Awater]_k = \exp(-bd),$$
where $d = 3 \cdot 10^{-3}$ $\text{mm}^2/\text{s}$ for water at $37^\circ$C. 

\ 

\noindent
For the \textbf{bi-tensor model} we assume that each voxel has a two
compartments. One compartment contains tissue and one contains free water. Let
$f$ be the fraction of the volume that is free water. Then we have 
$$[\Abitensor(\vect{D},f)]_k = 
  (1-f)[\Atissue(\vect{D})]_k +
  f [\Awater]_k$$
$f$ is a scalar that needs to be estimated for each voxel. $(0 \leq f < 1)$

\ 

\noindent
\textit{Note:} This is a departure from the notation in \cite{Pasternak2009}
which interchanges the values of $f$ and $1-f$. This is done so that we
maintain compatibility with the
\href{http://nipy.org/dipy/examples\_built/reconst\_fwdti.html}{multi-shell
free water elimination implementation} in \href{http://nipy.org/dipy}{DIPY}. 

\ 

\noindent
To fit the bi-tensor model we would like to estimate the best values of
$\vect{D}$ and $f$ that fit $$ [\hat{\vect{A}}]_k =
[\vect{A}_{\text{bi-tensor}}(\vect{D}), f)]_k$$ We need to estimate a scalar
$f$ and a real symmetric matrix $\vect{D}$ at each voxel. This is solvable if
$f=0$ everywhere and we are looking at a single compartment model. However, if
$f$ is a parameter to be estimated at each voxel then there are infinitely many
viable solutions. Choosing amongst them requires additional constraints. 

\section{Variational Framework}

%FIXME: Add background about framework

$$L(\vect{D}, f) = \int_\Omega \big( ||\vect{A}_{\text{bi-tensor}}(\vect{D}, f)
- \hat{\vect{A}}|| + \alpha \sqrt{|\gamma(\vect{D})|} \big) d\Omega$$
Here $\alpha$ is a parameter to control the regularization

%FIXME: Add constraints on volume fraction

\section{Induced Metric}

Instead of using the natural Riemannian metric on the spatial-feature space as
given in \cite{Pasternak2009}, we use the Euclidean metric given in
\cite{Pasternak2014} 

$$
h = 
\begin{pmatrix}
1 & & & & & & & & \\
 & 1 & & & & & & & \\
 & & 1 & & & & & & \\
 & & & \beta & & & & & \\
 & & & & \beta & & & & \\
 & & & & & \beta & & & \\
 & & & & & & \beta & & \\
 & & & & & & & \beta & \\
 & & & & & & & & \beta & 
\end{pmatrix}
$$
for some $\beta > 0$. Also the embedding is given by 
$$ \vect{x} = [x, y, z, D^{1 1}, D^{2 2}, D^{3 3}, \sqrt{2} D^{1 2}, \sqrt{2}
D^{2 3}, \sqrt{2} D^{1 3}].$$


\ 

\noindent
We can pull this metric back to get the induced metric. The components of the
induced metric are given in Einstein's summation notation by $$ \gamma_{\mu
\nu} = \partial_\mu \vect{x}^i \partial_\nu \vect{x}^j h_{ij}(\vect{x}).$$ The
indices $\mu$ and $\nu$ take values 1, 2 or 3 which correspond to the $x$, $y$,
and $z$ directions. The co-ordinates of the spatial-feature space are denoted
by $\vect{x}^i$ and the indices $i$ and $j$ take values 1 to 9.  Since $h$ is
diagonal $h_{i j}$ is only non-zero when $i = j$. So the double sum over $i$
and $j$ become a single sum over either one of the indices.  Expanding this out
we get 
$$\gamma_{\mu \nu} = 
\sum_{i=1}^3 \partial_\mu \vect{x}^i \partial_\nu \vect{x}^i + 
\beta \sum_{i=4}^9 \partial_\mu \vect{x}^i \partial_\nu \vect{x}^i$$

\ 

\noindent
The $\gamma$ matrix is symmetric so we only need to write out 6 equations. 

\begin{eqnarray*}
\gamma_{1 1} & = & 1 + \beta \sum_{i=4}^9 
  \big(\vect{x}^i_x\big)^2 \\
\gamma_{2 2} &=& 1 + \beta \sum_{i=4}^9 
  \big(\vect{x}^i_y\big)^2 \\
\gamma_{3 3} &=& 1 + \beta \sum_{i=4}^9 
  \big(\vect{x}^i_z\big)^2 \\
\gamma_{1 2} &=& \beta \sum_{i=4}^9 \vect{x}^i_x  \vect{x}^i_y  \\
\gamma_{2 3} &=& \beta \sum_{i=4}^9 \vect{x}^i_y \vect{x}^i_z  \\
\gamma_{1 3} &=& \beta \sum_{i=4}^9 \vect{x}^i_x \vect{x}^i_z  \\
\end{eqnarray*}
The subscripts $x$, $y$, $z$ denote partial derivatives w.r.t those co-ordinates.
We denote the determinant of the matrix $\gamma_{\mu \nu}$ by $|\gamma|$. We
also denote the inverse of the $\gamma_{\mu \nu}$ matrix by $\gamma^{\mu \nu}$.
Since the inverse of an invertible matrix is the transpose of the co-factor
matrix divided by the determinant, these two quantities are related by 
$$\gamma^{\mu \nu} = \frac{C_{\nu \mu}}{|\gamma|}$$
The cofactor $C_{\nu \mu}$ is computed by $C_{\nu \mu} = (-1)^{\nu + \mu}
M_{\nu \mu}$ where $M_{\nu \mu}$, a minor, is the determinant of the sub-matrix
of $\gamma$ with the $\nu$-th row and $\mu$-th column removed. Since $\gamma$
is symmetric, its inverse matrix is also symmetric. This means that we do not
need to take the transpose of the co-factor matrix and can also use the
identity $$\gamma^{\mu \nu} = \frac{C_{\mu \nu}}{|\gamma|}.$$

\begin{eqnarray*}
C_{1 1} &=&  (\gamma_{2 2} \gamma_{3 3} - \gamma_{2 3}^2) \\
C_{2 2} &=&  (\gamma_{1 1} \gamma_{3 3} - \gamma_{1 3}^2) \\
C_{3 3} &=&  (\gamma_{1 1} \gamma_{2 2} - \gamma_{1 2}^2) \\
C_{1 2} &=& (-\gamma_{1 2} \gamma_{3 3} + \gamma_{1 3} \gamma_{2 3}) \\
C_{2 3} &=& (-\gamma_{1 1} \gamma_{2 3} + \gamma_{1 3} \gamma_{1 2}) \\
C_{1 3} &=&  (\gamma_{1 2} \gamma_{2 3} - \gamma_{1 3} \gamma_{2 2}) 
\end{eqnarray*}
We also calculate the determinant as the expansion of the co-factors along the first row. 
$$|\gamma| = \gamma_{1 1} C_{1 1} + \gamma_{1 2} C_{1 2} + 
             \gamma_{1 3} C_{1 3}$$

\section{Gradient Descent}
The equations of motion in \cite{Pasternak2009}[A8] simplify with this new
choice of metric $h$ as the Christoffel numbers are zero. For the 6 tensor
elements $j \in {4, 5, \ldots, 9}$, 

$$\Delta\vect{x}^j = b\sum_{k=1}^n (\Ahat - \Abitensor) \Atissue
\bigg(\vect{q}_k^T\frac{\partial\vect{D}}{\partial\vx^j}\vect{q}_k\bigg) +
\frac{\alpha}{\sqdetgam} \partial_\mu (\sqdetgam \gamma^{\mu \nu} \partial_\nu 
\vx^j)
$$
where Einstein's summation notation is used in the second term on the right
hand side. For the fractional volume parameter we have

% FIXME: Since we have switched the values of f and 1-f in this document
% check to see what the sign needs to be on the equation below
$$\Delta f = -b \sum_{k=1}^n (\Ahat - \Abitensor)(\Atissue - \Awater)$$

\subsection{Beltrami Operator}
Here we write out the expression for the Beltrami operator $\Delta_\gamma$
$$\Delta_\gamma \vx^j = \frac{1}{\sqdetgam} \partial_\mu 
    (\sqdetgam \gamma^{\mu \nu} \partial_\nu \vx^j)$$
This expression is in Einstein Summation notation and $\mu$ and $\nu$ go
from 1 to 3. So we have 9 expressions that we consider 3 at a time. 

\ 

\noindent
Setting $\mu = 1$ we get three terms by letting $\nu$ go from 1 to 3.

\begin{eqnarray*}
& &\frac{1}{\sqdetgam} \frac{\partial}{\partial x}
\big(\sqdetgam \gamma^{11} \vx^j_x + 
 \sqdetgam \gamma^{12} \vx^j_y +
 \sqdetgam \gamma^{13} \vx^j_z\big) \\
&=&\frac{1}{\sqdetgam} \frac{\partial}{\partial x}
\bigg(\frac{C_{11}}{\sqdetgam} \vx^j_x + 
      \frac{C_{12}}{\sqdetgam} \vx^j_y +
      \frac{C_{13}}{\sqdetgam} \vx^j_z \bigg) \\
&=& \frac{1}{\sqdetgam} \pp{x} \bigg( \frac{p^j}{\sqdetgam} \bigg), 
      \quad\text{Setting}\quad 
      p^j = C_{11}\vx^j_x + C_{12}\vx^j_y + C_{13}\vx^j_z\\
&=& \frac{p^j_x}{\detgam} - \frac{p^j \gamma_x}{2\detgam^2}, 
      \quad\text{Using the quotient rule.}
\end{eqnarray*}
\noindent 
Setting $\mu = 2$ and $q^j = C_{12}\vx^j_x + C_{22}\vx^j_y + C_{23}\vx^j_z$, we get the next three terms as
$$ \frac{q^j_x}{\detgam} - \frac{q^j \gamma_x}{2\detgam^2}$$
\noindent 
Similarly, setting $\mu = 3$ and $r^j = C_{13}\vx^j_x + C_{23}\vx^j_y + C_{33}\vx^j_z$, we get the last three terms as
$$ \frac{r^j_x}{\detgam} - \frac{r^j \gamma_x}{2\detgam^2}$$

\noindent \textit{Note}: The definition of $p^j$ and $q^j$ is a generalization
of the ones given in the discretization of the Beltrami flow given in
\cite{Kimmel} (Equations 10.4 and 10.5).

\ 

\noindent
Putting it all together
$$\Delta_\gamma \vx^j = \frac{1}{\detgam}(p^j_x + q^j_y + r^j_z) - 
\frac{1}{2\detgam^2}(p^j \gamma_x + q^j \gamma_y + r^j \gamma_z).$$






\begin{thebibliography}{9}
        \bibitem[Pasternak2009]{Pasternak2009} Pasternak, O. , Sochen, N. ,
                Gur, Y. , Intrator, N. and Assaf, Y. (2009), \textit{Free water
                elimination and mapping from diffusion MRI.} Magn. Reson.
                Med., 62: 717-730.
                \href{https://doi.org/10.1002/mrm.22055}{doi:10.1002/mrm.22055}

        \bibitem[Pasternak2014]{Pasternak2014} Pasternak O., Maier-Hein K.,
                Baumgartner C., Shenton M.E., Rathi Y., Westin CF.  (2014)
                \textit{The Estimation of Free-Water Corrected Diffusion
                Tensors.} In: Westin CF., Vilanova A., Burgeth B. (eds)
                Visualization and Processing of Tensors and Higher Order
                Descriptors for Multi-Valued Data. Mathematics and
                Visualization.  Springer, Berlin, Heidelberg
                \href{https://doi.org/10.1007/978-3-642-54301-2\_11}{doi:10.1007/978-3-642-54301-2\_11}

        \bibitem[Kimmel]{Kimmel} Kimmel, Ron. \textit{Geometric Framework in
                Image Processing.} Numerical Geometry of Images: Theory,
                Algorithms, and Applications. N.p.: Springer, 2012. N. pag.
                Print.
                \href{https://doi.org/10.1007/978-0-387-21637-9}{doi:10.1007/978-0-387-21637-9}

\end{thebibliography}


\end{document}
